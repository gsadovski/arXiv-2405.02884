\documentclass[../main.tex]{subfiles}

\begin{document}

\section{Algebraic renormalizability}%
\label{sec:quantum}

To prove the renormalizability of the topological symmetry-restaured phase of gravity, we follow the algebraic renormalization program~\cite{piguet1995b}. The results obtained are valid to all orders in perturbation theory, and are independent of any regularization scheme.

First, we fix its gauge symmetry by adding the (A)SDL gauge fixing action,
\begin{subequations}%
  \label{eq:tg-gf-action}
  \begin{align}
    S_{\text{GF}} & = s \int \tr \left[ \bar{c} d \star \omega + \bar{\phi} d \star \psi + \bar{\chi} \left( R \pm \star R \right) \right] \;,                                                                                                                  \\
                  & = \int \tr \left[ b d \star \omega - \bar{c} d \star Dc + \left( \bar{c} + \bar{\eta} + \left[ c,\bar{\phi} \right] \right) d \star \psi + \bar{\phi} d \star D \phi + d c \left[ \star \psi ,\bar{\phi} \right] + \right. \nonumber        \\
                  & + \left. \left( B + \left[ c,\bar{\chi} \right] \right) \left( R \pm \star R \right) + \bar{\chi} \left( D \pm \star D \right) \psi \vphantom{ \left( \bar{c} + \bar{\eta} + \left[ c,\bar{\phi} \right] \right) d \star \psi } \right] \;.
  \end{align}
\end{subequations}

Second, the non-linearity of the BRST, and of the \textquote{non-linear bosonic ghost symmetry}~\eqref{eq:nl-bosonic-symmetry}, will inevitably appear as insertions in the correlation function of the quantum theory. To account for these infinities, we need to explicitly include them as $ s $-boundaries. Consider the $ s $-doublets,
\begin{subequations}%
  \label{eq:nl-s-doublets}
  \begin{align}
    s\tau    & = \Omega \;, \\
    s\Omega  & = 0 \;,      \\
    sE       & = L \;,      \\
    sL       & = 0 \;,      \\
    s\lambda & = K \;,      \\
    sK       & = 0 \;,      \\
    sZ       & = H \;,      \\
    sH       & = 0 \;.
    \;
  \end{align}
\end{subequations}
We define the non-linearity action functional as
\begin{subequations}%
  \label{eq:nl-action}
  \begin{align}
    S_{\text{NL}} & = s \int \tr \left( \tau Dc + E c^2 + \lambda \left[ c,\bar\chi \right] + Z \left[ c,Y \right] \right) \;, \label{eq:nl-action-1}                                                                                           \\
                  & = \int \tr \left\{ \Omega Dc + Lc^2 + \tau \left( D\phi + \left[ c,\psi \right] \right) + E \left[ c,\phi \right] + K \left[ c,\bar\chi \right] + \lambda \left(\vphantom{^2}\left[ c,B \right] + \right. \right. \nonumber \\
                  & + \left. \left. \left[ c^2,\bar\chi \right] + \left[ \bar\chi,\phi \right] \right) \right\} \;.
  \end{align}
\end{subequations}
The grading of the newly introduced fields can be found at~\ref{tab:nl-sources}. Thus, the full classical action to be consider is given by
\begin{equation}
  \label{eq:total-action}
  \Sigma \equiv S + S_{\text{GF}} + S_{ \text{NL} }\; .
\end{equation}

\begin{table}[htpb]
  \caption{Grading of Zwanzinger sources for symmetry non-linearities.}%
  \label{tab:nl-sources}
  \begin{tabular}{ccccccccc}
    \toprule
    Field      & $\tau$ & $\Omega$ & $E$ & $L$  & $\Lambda$ & $K$  & $Z$ & $H$  \\
    \midrule
    Form rank  & 3      & 3        & 4   & 4    & 2         & 2    & 2   & 2    \\
    Ghost no.  & -2     & -1       & -3  & -2   & -1        & 0    & -1  & 0    \\
    Statistics & odd    & even     & odd & even & odd       & even & odd & even \\
    \bottomrule
  \end{tabular}
\end{table}

\subsubsection{Ward identities}%
\label{ssec:ward_identities}
%
The local symmetries of $ \Sigma $, including linearly broken ones, are:
\begin{itemize}
  \item Traditional gauge fixing equation
        \begin{equation}
          \label{eq:gfeq}
          \frac{ \delta \Sigma }{ \delta b } = d \star A \;;
        \end{equation}
  \item The topological gauge fixing equation
        \begin{equation}
          \label{eq:topgfeq}
          \frac{\delta\Sigma}{\delta\bar{\eta}} = d\star{\psi} \;;
        \end{equation}
  \item The Faddeev-Popov anti-ghost equation
        \begin{equation}
          \label{eq:fpantighosteq}
          \mathcal{G}_{\bar{c}} \left(\Sigma\right)=d\star{\psi} \;,
        \end{equation}
        where
        \begin{equation}
          \label{eq:fpantighostop}
          \mathcal{G}_{\bar{c}}\equiv \frac{\delta\phantom{c}}{\delta\bar{c}}+d\star{\frac{\delta\phantom{\Omega}}{\delta\Omega}} \;;
        \end{equation}
  \item The bosonic anti-ghost equation
        \begin{equation}
          \label{eq:bosonicantighosteq}
          \mathcal{G}_{\bar{\phi}} \left(\Sigma\right)=0\;,
        \end{equation}
        where
        \begin{equation}
          \label{eq:bosonicantighostop}
          \mathcal{G}_{\bar{\phi}}\equiv \frac{\delta}{\delta\bar{\phi}}-d\star{\frac{\delta}{\delta\tau}}\;.
        \end{equation}
\end{itemize}
The equations~\eqref{eq:topgfeq} and~\eqref{eq:fpantighosteq} can be combined into an exact symmetry of $ \Sigma $. Additionally, the global symmetries of $ \Sigma $, including linearly broken ones, are:
\begin{itemize}
  \item  The Slavnov-Taylor identity
        \begin{equation}
          \label{eq:st-identity}
          \mathcal{S}\left(\Sigma\right)=0 \;,
        \end{equation}
        where
        \begin{align}
          \label{eq:SToperator}
          \mathcal{S} & \equiv \int \tr\left[\left(\psi-\frac{\delta\phantom{\Omega}}{\delta\Omega}\right)\frac{\delta\phantom{A}}{\delta A}+\left(\phi-\frac{\delta\phantom{L}}{\delta L}\right)\frac{\delta\phantom{c}}{\delta c}-\frac{\delta\phantom{\tau}}{\delta\tau}\frac{\delta\phantom{\psi}}{\delta\psi}-\frac{\delta\phantom{E}}{\delta E}\frac{\delta\phantom{\phi}}{\delta\phi}+X\frac{\delta\phantom{Y}}{\delta Y} \right. + \nonumber \\
                      & + \left. b\frac{\delta\phantom{\bar{c}}}{\delta\bar{c}}+B\frac{\delta\phantom{\bar{\chi}}}{\delta\bar{\chi}}+\bar{\eta}\frac{\delta\phantom{\bar{\phi}}}{\delta\bar{\phi}} + \Omega \frac{\delta\phantom{\tau}}{\delta\tau}+L\frac{\delta\phantom{E}}{\delta E}+K\frac{\delta\phantom{\lambda}}{\delta\lambda}+H\frac{\delta\phantom{Z}}{\delta Z}\right] \;;
        \end{align}
  \item The bosonic anti-ghost equation
        \begin{equation}
          \mathcal{G}_\phi \left(\Sigma\right) = \Delta_\phi \;,
        \end{equation}
        where
        \begin{equation}
          \mathcal{G}_\phi \equiv \int \left(\frac{\delta\phantom{\phi}}{\delta\phi}-\left[\bar{\phi},\frac{\delta\phantom{b}}{\delta b}\right]\right) \;,
        \end{equation}
        and
        \begin{equation}
          \label{eq:linearbreakphi}
          \Delta_\phi \equiv \int \left(\left[A,\tau\right]+\left[c,E\right]+\left[\bar{\chi},\lambda\right]+\left[Y,Z\right]\right) \;;
        \end{equation}
  \item The 1st FP ghost equation
        \begin{equation}
          \label{eq:1stfpghosteq}
          \mathcal{G}^{\left(1\right)}_c \left(\Sigma\right)=\Delta_c \;,
        \end{equation}
        where
        \begin{align}
          \label{eq:1stfpghostop}
          \mathcal{G}^{\left(1\right)}_c & \equiv \int \left(\frac{\delta\phantom{c}}{\delta c}-\left[Y,\frac{\delta\phantom{X}}{\delta X}\right]-\left[\bar{c},\frac{\delta\phantom{b}}{\delta b}\right]-\left[\bar{\phi},\frac{\delta\phantom{\bar{\eta}}}{\delta\bar{\eta}}\right]-\left[\bar{\chi},\frac{\delta\phantom{B}}{\delta B}\right]-\left[\lambda,\frac{\delta\phantom{K}}{\delta K}\right] + \left. \nonumber \\
                                         & \left. - \left[Z,\frac{\delta\phantom{H}}{\delta H}\right]\right) \;,
        \end{align}
        and
        \begin{align}
          \label{eq:linearbreakc}
          \Delta_c & \equiv \int \left(\left[A,\Omega\right]+\left[L,c\right]+\left[\tau,\psi\right]+\left[E,\phi\right]+\left[\bar{\chi},K\right]+\left[\lambda,B\right] + \left[Y,H\right] \right. \nonumber \\
                   & + \left. \left[Z,X\right] \right) \;;
        \end{align}
  \item The 2nd Faddeev-Popov ghost equation
        \begin{equation}
          \label{eq:2ndfpghosteq}
          \mathcal{G}^{\left(2\right)}_c \left(\Sigma\right)=\Delta_c \;,
        \end{equation}
        where
        \begin{equation}
          \label{eq:2ndfpghostop}
          \mathcal{G}^{\left(2\right)}_c \equiv \int \left(\frac{\delta\phantom{c}}{\delta c}+\left[A,\frac{\delta\phantom{\psi}}{\delta \psi}\right]+\left[c,\frac{\delta\phantom{\phi}}{\delta \phi}\right]-\left[\bar{\phi},\frac{\delta\phantom{\bar{c}}}{\delta\bar{c}}\right]+\left[\tau,\frac{\delta\phantom{\Omega}}{\delta \Omega}\right]+\left[E,\frac{\delta\phantom{L}}{\delta L}\right]\right)\;;
        \end{equation}
  \item  The vectorial supersymmetry
        \begin{equation}
          \label{eq:vecsusyeq}
          \mathcal{W}\left(\Sigma\right)=0 \;,
        \end{equation}
        where
        \begin{align}
          \label{eq:vecsusyop}
          \mathcal{W} & \equiv \int \tr \left[\mathcal{L}_\xi A\frac{\delta\phantom{\psi}}{\delta\psi}+\mathcal{L}_\xi c\frac{\delta\phantom{\phi}}{\delta\phi}+\mathcal{L}_\xi Y \frac{\delta\phantom{X}}{\delta X}-\mathcal{L}_\xi\left(\bar{c}+\bar{\eta}\right)\frac{\delta\phantom{b}}{\delta b}+\mathcal{L}_\xi \bar{\chi}\frac{\delta\phantom{B}}{\delta B}+ \right. \nonumber \\
                      & + \left. \mathcal{L}_\xi\bar{\phi}\left(\frac{\delta\phantom{\bar{c}}}{\delta\bar{c}}-\frac{\delta\phantom{\bar{\eta}}}{\delta\bar{\eta}}\right) + \mathcal{L}_\xi\tau\frac{\delta\phantom{\Omega}}{\delta\Omega}+\mathcal{L}_\xi E\frac{\delta\phantom{L}}{\delta L}+\mathcal{L}_\xi \lambda\frac{\delta\phantom{K}}{\delta K}\right]
        \end{align}
  \item  The non-linear bosonic symmetry is
        \begin{equation}
          \label{eq:t-symmetry}
          \mathcal{T}\left(\Sigma\right)=0\;,
          \;
        \end{equation}
        where
        \begin{equation}
          \label{eq:t-operator}
          \mathcal{T} \equiv \int \tr \left[\frac{\delta\phantom{\Omega}}{\delta \Omega}\frac{\delta\phantom{\psi}}{\delta \psi} + \frac{\delta\phantom{H}}{\delta H}\frac{\delta\phantom{X}}{\delta X}+\frac{\delta\phantom{L}}{\delta L}\frac{\delta\phantom{\phi}}{\delta \phi}+\frac{\delta\phantom{K}}{\delta K}\frac{\delta\phantom{B}}{\delta B}+\left(\bar{c}+\bar{\eta}\right)\left(\frac{\delta\phantom{\bar{c}}}{\delta\bar{c}}-\frac{\delta\phantom{\bar{\eta}}}{\delta\bar{\eta}}\right)\right] \;;
        \end{equation}
  \item And, finally, the fermionic ghost symmetry is
        \begin{equation}
          \label{eq:fermionicghostsymmetry}
          \mathcal{F}\left(\Sigma\right)=0 \;,
        \end{equation}
        where
        \begin{equation}
          \label{eq:fermionicghostop}
          \mathcal{F} \equiv \varint \mathrm{Tr} \left[c\frac{\delta\phantom{\phi}}{\delta \phi}+\bar{\phi}\left(\frac{\delta\phantom{\bar{c}}}{\delta\bar{c}}-\frac{\delta\phantom{\bar{\eta}}}{\delta\bar{\eta}}\right)-\tau\frac{\delta\phantom{\Omega}}{\delta \Omega}-2E\frac{\delta\phantom{L}}{\delta L}-\lambda\frac{\delta\phantom{K}}{\delta K}\right] \;
        \end{equation}

\end{itemize}

%\subsubsection{The most general counterterm}%
%\label{ssub:the_most_general_counterterm}
%\begin{equation}
%  \label{eq:quantumaction}
%  \Gamma^{(1)}=\Sigma+\epsilon \Sigma^{\text{ct}} \;
%\end{equation}
%
%where
%\begin{align}
%  \mathcal{S}_\Sigma & \equiv \varint\mathrm{Tr}\left[\left(\psi-\frac{\delta\Sigma}{\delta\Omega}\right)\frac{\delta\phantom{A}}{\delta A}+\frac{\delta\Sigma}{\delta A}\frac{\delta\phantom{\Omega}}{\delta \Omega}+\left(\phi-\frac{\delta\Sigma}{\delta L}\right)\frac{\delta\phantom{c}}{\delta c}+\frac{\delta\Sigma}{\delta c}\frac{\delta\phantom{L}}{\delta L}-\frac{\delta\Sigma}{\delta\tau}\frac{\delta\phantom{\psi}}{\delta\psi}+\frac{\delta\Sigma}{\delta \psi}\frac{\delta\phantom{\tau}}{\delta \tau}-\frac{\delta\Sigma}{\delta E}\frac{\delta\phantom{\phi}}{\delta\phi}\;\right.+\nonumber \\
%                     & + \left. \frac{\delta\Sigma}{\delta \phi}\frac{\delta\phantom{E}}{\delta E}+X\frac{\delta\phantom{Y}}{\delta Y}+b\frac{\delta\phantom{\bar{c}}}{\delta\bar{c}}+B\frac{\delta\phantom{\bar{\chi}}}{\delta\bar{\chi}}+\bar{\eta}\frac{\delta\phantom{\bar{\phi}}}{\delta\bar{\phi}}+\Omega\frac{\delta\phantom{\tau}}{\delta\tau}+L\frac{\delta\phantom{E}}{\delta E}+K\frac{\delta\phantom{\lambda}}{\delta\lambda}+H\frac{\delta\phantom{Z}}{\delta Z}\right]\;,
%\end{align}
%and
%\begin{align}
%  \mathcal{T}_\Sigma & \equiv \varint \mathrm{Tr} \left[\frac{\delta\Sigma}{\delta \Omega}\frac{\delta\phantom{\psi}}{\delta \psi}-\frac{\delta\Sigma}{\delta\psi}\frac{\delta\phantom{\Omega}}{\delta \Omega} + \frac{\delta\Sigma}{\delta H}\frac{\delta\phantom{X}}{\delta X}-\frac{\delta\Sigma}{\delta X}\frac{\delta\phantom{H}}{\delta H}+\frac{\delta\Sigma}{\delta L}\frac{\delta\phantom{\phi}}{\delta \phi}-\frac{\delta\Sigma}{\delta \phi}\frac{\delta\phantom{L}}{\delta L}+\frac{\delta\Sigma}{\delta K}\frac{\delta\phantom{B}}{\delta B}-\frac{\delta\Sigma}{\delta B}\frac{\delta\phantom{K}}{\delta K}\right.+\;\nonumber \\
%                     & +\left.\left(\bar{c}+\bar{\eta}\right)\left(\frac{\delta\phantom{\bar{c}}}{\delta\bar{c}}-\frac{\delta\phantom{\bar{\eta}}}{\delta\bar{\eta}}\right)\right]
%\end{align}
%are the linearized versions of \eqref{eq:SToperator} and \eqref{eq:t-operator}, respectively.
%
\subsection{Quantum stability}\label{sec:stability;sec:quantum}

\end{document}
