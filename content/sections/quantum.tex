\documentclass[../main.tex]{subfiles}

\begin{document}

\section{Algebraic renormalizability}%
\label{sec:quantum}

To prove the renormalizability of the proposed topological symmetry-restored phase of gravity, we follow the algebraic renormalization program~\cite{piguet1995b}. The results obtained are valid to all orders in perturbation theory, and are independent of any regularization scheme.

First, we fix the gauge symmetry of~\eqref{eq:tg+sb} by adding to it the extended (A){}SDL gauge fixing action,
\begin{subequations}%
  \label{eq:tg-gf-action}
  \begin{align}
    S_{\text{GF}} & = s \int \tr \left[ \bar{c} d \star \omega + \bar{\phi} d \star \psi + \bar{\chi} R^{\pm} \right] \;,                                                                                                                                \\
                  & = \int \tr \left[ b d \star \omega - \bar{c} d \star Dc + \left( \bar{c} + \bar{\eta} + \left[ c,\bar{\phi} \right] \right) d \star \psi + \bar{\phi} d \star D \phi + d c \left[ \star \psi ,\bar{\phi} \right] + \right. \nonumber \\
                  & + \left. \left( B + \left[ c,\bar{\chi} \right] \right) R^{\pm} + \bar{\chi} {\left( D \psi \right)}^{\pm} \right] \;.
  \end{align}
\end{subequations}

Second, the non-linearity of the BRST transformations, and of the non-linear bosonic ghost symmetry,~\eqref{eq:nl-bosonic-eq}, will inevitably appear as insertions in the correlation function of the quantized theory. To account for these infinities, we need to explicitly include them in the total action. Consider the $ s $-doublets,
\begin{subequations}%
  \label{eq:nl-s-doublets}
  \begin{align}
    s\tau    & = \Omega \;, \;\; s\Omega   = 0 \;, \\
    sE       & = L \;, \;\; sL        = 0 \;,      \\
    s\lambda & = K \;, \;\; sK        = 0 \;,      \\
    sZ       & = H \;, \;\; sH        = 0 \;.
  \end{align}
\end{subequations}
The non-linearity action functional
\begin{subequations}%
  \label{eq:nl-action}
  \begin{align}
    S_{\text{NL}} & = s \int \tr \left( \tau Dc + E c^2 + \lambda \left[ c,\bar\chi \right] + Z \left[ c,Y \right] \right) \;, \label{eq:nl-action-1}                                                                                           \\
                  & = \int \tr \left\{ \Omega Dc + Lc^2 + \tau \left( D\phi + \left[ c,\psi \right] \right) + E \left[ c,\phi \right] + K \left[ c,\bar\chi \right] + \lambda \left(\vphantom{^2}\left[ c,B \right] + \right. \right. \nonumber \\
                  & + \left. \left. \left[ c^2,\bar\chi \right] + \left[ \bar\chi,\phi \right] \right) + H \left[ c, Y \right] + Z \left( \left[ Y, \phi \right] + \left[ c^2, Y \right] + \left[ c, X \right] \right) \right\} \;,
  \end{align}
\end{subequations}
is the most general local, power-counting renormalizable, invariant polynomial that explicitly couples all independent non-linearities to external sources, while remaining an $ s $-boundary. The grading of the newly introduced fields can be found at Table~\ref{tab:nl-sources}.

The full classical action to be considered is
\begin{equation}
  \label{eq:total-action}
  \Sigma \equiv S + S_{\text{GF}} + S_{ \text{NL} }\; .
\end{equation}

\begin{table}[htpb]
  \caption{Grading of Zwanzinger sources for symmetry non-linearities.}%
  \label{tab:nl-sources}
  \begin{tabular}{ccccccccc}
    \toprule
    Field      & $\tau$ & $\Omega$ & $E$ & $L$  & $\lambda$ & $K$  & $Z$ & $H$  \\
    \midrule
    Form rank  & 3      & 3        & 4   & 4    & 2         & 2    & 2   & 2    \\
    Ghost no.  & -2     & -1       & -3  & -2   & -1        & 0    & -1  & 0    \\
    Statistics & odd    & even     & odd & even & odd       & even & odd & even \\
    \bottomrule
  \end{tabular}
\end{table}

\subsection{Ward identities}%
\label{ssec:ward_identities}
%
The local symmetries of $ \Sigma $, including linearly broken ones, are:
\begin{itemize}
  \item Traditional gauge fixing equation
        \begin{equation}
          \label{eq:gfeq}
          \frac{ \delta \Sigma }{ \delta b } = d \star \omega  \;;
        \end{equation}
  \item The topological gauge fixing equation
        \begin{equation}
          \label{eq:topgfeq}
          \frac{\delta\Sigma}{\delta\bar{\eta}} = d\star{\psi} \;;
        \end{equation}
  \item The Faddeev-Popov anti-ghost equation
        \begin{equation}
          \label{eq:fpantighosteq}
          \mathcal{G}_{\bar{c}} \left(\Sigma\right)=d\star{\psi} \;,
        \end{equation}
        where
        \begin{equation}
          \label{eq:fpantighostop}
          \mathcal{G}_{\bar{c}}\equiv \frac{\delta\phantom{c}}{\delta\bar{c}}+d\star{\frac{\delta\phantom{\Omega}}{\delta\Omega}} \;;
        \end{equation}
  \item The bosonic anti-ghost equation
        \begin{equation}
          \label{eq:bosonicantighosteq}
          \mathcal{G}_{\bar{\phi}} \left(\Sigma\right)=0\;,
        \end{equation}
        where
        \begin{equation}
          \label{eq:bosonicantighostop}
          \mathcal{G}_{\bar{\phi}}\equiv \frac{\delta}{\delta\bar{\phi}}-d\star{\frac{\delta}{\delta\tau}}\;.
        \end{equation}
\end{itemize}
The equations~\eqref{eq:topgfeq} and~\eqref{eq:fpantighosteq} can be combined into an exact symmetry of $ \Sigma $. Additionally, the global symmetries of $ \Sigma $, including linearly broken ones, are:
\begin{itemize}
  \item  The Slavnov-Taylor identity
        \begin{equation}
          \label{eq:st-identity}
          \mathcal{S}\left(\Sigma\right)=0 \;,
        \end{equation}
        where
        \begin{align}
          \label{eq:SToperator}
          \mathcal{S} & \equiv \int \tr\left[\left(\psi-\frac{\delta\phantom{\Omega}}{\delta\Omega}\right)\frac{\delta\phantom{\omega }}{\delta \omega }+\left(\phi-\frac{\delta\phantom{L}}{\delta L}\right)\frac{\delta\phantom{c}}{\delta c}-\frac{\delta\phantom{\tau}}{\delta\tau}\frac{\delta\phantom{\psi}}{\delta\psi}-\frac{\delta\phantom{E}}{\delta E}\frac{\delta\phantom{\phi}}{\delta\phi}+X\frac{\delta\phantom{Y}}{\delta Y} \right. + \nonumber \\
                      & + \left. b\frac{\delta\phantom{\bar{c}}}{\delta\bar{c}}+B\frac{\delta\phantom{\bar{\chi}}}{\delta\bar{\chi}}+\bar{\eta}\frac{\delta\phantom{\bar{\phi}}}{\delta\bar{\phi}} + \Omega \frac{\delta\phantom{\tau}}{\delta\tau}+L\frac{\delta\phantom{E}}{\delta E}+K\frac{\delta\phantom{\lambda}}{\delta\lambda}+H\frac{\delta\phantom{Z}}{\delta Z}\right] \;;
        \end{align}
  \item The 1st FP ghost equation
        \begin{equation}
          \label{eq:1stfpghosteq}
          \mathcal{G}^{\left(1\right)}_c \left(\Sigma\right)=\Delta_c \;,
        \end{equation}
        where
        \begin{align}
          \label{eq:1stfpghostop}
          \mathcal{G}^{\left(1\right)}_c & \equiv \int \left(\frac{\delta\phantom{c}}{\delta c}-\left[Y,\frac{\delta\phantom{X}}{\delta X}\right]-\left[\bar{c},\frac{\delta\phantom{b}}{\delta b}\right]-\left[\bar{\phi},\frac{\delta\phantom{\bar{\eta}}}{\delta\bar{\eta}}\right]-\left[\bar{\chi},\frac{\delta\phantom{B}}{\delta B}\right]-\left[\lambda,\frac{\delta\phantom{K}}{\delta K}\right] + \right. \nonumber \\
                                         & \left. - \left[Z,\frac{\delta\phantom{H}}{\delta H}\right]\right) \;,
        \end{align}
        and
        \begin{align}
          \label{eq:linearbreakc}
          \Delta_c & \equiv \int \left(\left[\omega ,\Omega\right]+\left[L,c\right]+\left[\tau,\psi\right]+\left[E,\phi\right]+\left[\bar{\chi},K\right]+\left[\lambda,B\right] + \left[Y,H\right] \right. \nonumber \\
                   & + \left. \left[Z,X\right] \right) \;;
        \end{align}
  \item The 2nd Faddeev-Popov ghost equation
        \begin{equation}
          \label{eq:2ndfpghosteq}
          \mathcal{G}^{\left(2\right)}_c \left(\Sigma\right)=\Delta_c \;,
        \end{equation}
        where
        \begin{equation}
          \label{eq:2ndfpghostop}
          \mathcal{G}^{\left(2\right)}_c \equiv \int \left(\frac{\delta\phantom{c}}{\delta c}+\left[\omega ,\frac{\delta\phantom{\psi}}{\delta \psi}\right]+\left[c,\frac{\delta\phantom{\phi}}{\delta \phi}\right]-\left[\bar{\phi},\frac{\delta\phantom{\bar{c}}}{\delta\bar{c}}\right]+\left[\tau,\frac{\delta\phantom{\Omega}}{\delta \Omega}\right]+\left[E,\frac{\delta\phantom{L}}{\delta L}\right]\right)\;;
        \end{equation}
  \item The bosonic ghost equation
        \begin{equation}
          \mathcal{G}_\phi \left(\Sigma\right) = \Delta_\phi \;,
        \end{equation}
        where
        \begin{equation}
          \mathcal{G}_\phi \equiv \int \left(\frac{\delta\phantom{\phi}}{\delta\phi}-\left[\bar{\phi},\frac{\delta\phantom{b}}{\delta b}\right]\right) \;,
        \end{equation}
        and
        \begin{equation}
          \label{eq:linearbreakphi}
          \Delta_\phi \equiv \int \left(\left[\omega ,\tau\right]+\left[c,E\right]+\left[\bar{\chi},\lambda\right]+\left[Y,Z\right]\right) \;;
        \end{equation}
  \item  The vectorial supersymmetry
        \begin{equation}
          \label{eq:vecsusyeq}
          \mathcal{W}\left(\Sigma\right)=0 \;,
        \end{equation}
        where
        \begin{align}
          \label{eq:vecsusyop}
          \mathcal{W} & \equiv \int \tr \left[\mathcal{L}_\xi \omega \frac{\delta\phantom{\psi}}{\delta\psi}+\mathcal{L}_\xi c\frac{\delta\phantom{\phi}}{\delta\phi} - \mathcal{L}_\xi Y \frac{\delta\phantom{X}}{\delta X}-\mathcal{L}_\xi\left(\bar{c}+\bar{\eta}\right)\frac{\delta\phantom{b}}{\delta b} - \mathcal{L}_\xi \bar{\chi}\frac{\delta\phantom{B}}{\delta B}+ \right. \nonumber \\
                      & + \left. \mathcal{L}_\xi\bar{\phi}\left(\frac{\delta\phantom{\bar{c}}}{\delta\bar{c}}-\frac{\delta\phantom{\bar{\eta}}}{\delta\bar{\eta}}\right) + \mathcal{L}_\xi\tau\frac{\delta\phantom{\Omega}}{\delta\Omega}+\mathcal{L}_\xi E\frac{\delta\phantom{L}}{\delta L}+\mathcal{L}_\xi \lambda\frac{\delta\phantom{K}}{\delta K}\right] \;,
        \end{align}
        and $ \mathcal{L}_{\xi} $ is the Lie derivative along the vector field $ \xi $. It is assumed that $\xi$ is a Killing vector, and that it generates a flow of diffeomorphism on spacetime\footnote{Consequently, $\int \mathcal{L}_{\xi} \left( \varphi \star \mathds{1} \right) = 0 \;, \forall \; \varphi \in C^{ \infty }\left( \mathbb{R}^4 \right)$, and $ \star \mathcal{L}_\xi = \mathcal{L}_\xi \star $.}.
  \item  The non-linear bosonic symmetry
        \begin{equation}
          \label{eq:nl-bosonic-eq}
          \mathcal{T}\left(\Sigma\right)=0\;,
          \;
        \end{equation}
        where
        \begin{equation}
          \label{eq:t-operator}
          \mathcal{T} \equiv \int \tr \left[\frac{\delta\phantom{\Omega}}{\delta \Omega}\frac{\delta\phantom{\psi}}{\delta \psi} + \frac{\delta\phantom{H}}{\delta H}\frac{\delta\phantom{X}}{\delta X}+\frac{\delta\phantom{L}}{\delta L}\frac{\delta\phantom{\phi}}{\delta \phi}+\frac{\delta\phantom{K}}{\delta K}\frac{\delta\phantom{B}}{\delta B}+\left(\bar{c}+\bar{\eta}\right)\left(\frac{\delta\phantom{\bar{c}}}{\delta\bar{c}}-\frac{\delta\phantom{\bar{\eta}}}{\delta\bar{\eta}}\right)\right] \;;
        \end{equation}
  \item And, finally, the fermionic ghost symmetry
        \begin{equation}
          \label{eq:fermionicghostsymmetry}
          \mathcal{F}\left(\Sigma\right)=0 \;,
        \end{equation}
        where
        \begin{equation}
          \label{eq:fermionicghostop}
          \mathcal{F} \equiv \int \tr \left[c\frac{\delta\phantom{\phi}}{\delta \phi}+\bar{\phi}\left(\frac{\delta\phantom{\bar{c}}}{\delta\bar{c}}-\frac{\delta\phantom{\bar{\eta}}}{\delta\bar{\eta}}\right)-\tau\frac{\delta\phantom{\Omega}}{\delta \Omega}-2E\frac{\delta\phantom{L}}{\delta L}-\lambda\frac{\delta\phantom{K}}{\delta K}\right] \;.
        \end{equation}
\end{itemize}
Equations~\eqref{eq:1stfpghosteq} and~\eqref{eq:2ndfpghosteq} can also be joined to form an exact symmetry of $ \Sigma  $. The vectorial supersymmetry~\eqref{eq:vecsusyeq} is present in several topological field theories. It is a very strong symmetry, responsible for the 1-loop exactness of $ n=3 $ Chern-Simons theory~\cite{blasi1991a,maggiore1992a,piguet1995a}, and the tree-level exactness of TYM theory~\cite{brandhuber1994a,sadovski2017c,sadovski2018a}, for instance. The non-linear bosonic~\eqref{eq:nl-bosonic-eq}, and the fermionic ghost symmetry~\eqref{eq:fermionicghostsymmetry}, first reported by the authors (and collaborators) in QTYM~\cite{sadovski2017c}, are also present here, and are known to drastically reduce the number of independent renormalizations.

\subsection{Counterterms}%
\label{ssec:counterterm}

The Quantum Action Principle establishes the formal relationship between the set of Ward identities for $ \Sigma $, and the set of Ward identities for its associated quantum effective action, $ \Gamma $. The list above translates to the following set of symmetries for the quantum symmetry-restored phase of gravity:
\begin{subequations}%
  \label{eq:ward-identities}
  \begin{align}
    \frac{ \delta \Sigma^{ \left( n \right) } }{ \delta b }            & = 0 \;,                                \\
    \frac{ \delta \Sigma^{ \left( n \right) } }{ \delta \bar{ \eta } } & = 0 \;,                                \\
    \mathcal{G}_{ \bar{ c } } \left( \Sigma ^{ (n) } \right)           & = 0 \;,                                \\
    \mathcal{G}_{ \bar{ \phi  } } \left( \Sigma ^{ (n) } \right)       & = 0 \;,                                \\
    \frac{ \delta \Sigma^{ \left( n \right) } }{ \delta \bar{ \eta } } & = 0 \;,                                \\
    \mathcal{S}_{\Gamma^{(n-1)}} \left( \Sigma^{ (n)} \right)          & = 0 \;, \label{eq:quantum-st-identity} \\
    \mathcal{G}_{ \phi } \left( \Sigma ^{ (n) } \right)                & = 0 \;,                                \\
    \mathcal{G}^{(1)}_{ c } \left( \Sigma ^{ (n) } \right)             & = 0 \;,                                \\
    \mathcal{G}^{(2)}_{ c } \left( \Sigma ^{ (n) } \right)             & = 0 \;,                                \\
    \mathcal{W} \left( \Sigma^{ (n)} \right)                           & = 0 \;,                                \\
    \mathcal{T}_{\Gamma^{(n-1)}} \left( \Sigma^{ (n)} \right)          & = 0 \;,                                \\
    \mathcal{F} \left( \Sigma^{ (n)} \right)                           & = 0 \;,
  \end{align}
\end{subequations}
where $\Gamma^{(n)} \equiv \Sigma + \epsilon \Sigma^{(1)} + \cdots + \Sigma^{ (n) }$ is $ \Gamma  $ truncated at $ n $-th loop order, $ \epsilon $ is a small perturbative parameter, and $ \Sigma^{ (n)} $ is the $n$-loop radiative correction to $ \Sigma  $. The linear breaks are gone, and the non-linear operators $ \mathcal{S} $ and $ \mathcal{T} $ are replaced by their linearized versions
\begin{align}
  \label{eq:linear-st-operator}
  \mathcal{S}_{\Gamma^{ (n-1) }} & \equiv \int \tr \left[\left(\psi-\frac{\delta\Gamma^{ (n-1) }}{\delta\Omega}\right)\frac{\delta\phantom{\omega}}{\delta \omega}-\frac{\delta\Gamma^{ (n-1) }}{\delta \omega}\frac{\delta\phantom{\Omega}}{\delta \Omega}+\left(\phi-\frac{\delta\Gamma^{ (n-1) }}{\delta L}\right)\frac{\delta\phantom{c}}{\delta c} \right. + \nonumber                                                                                                 \\
                                 & - \left. \frac{\delta\Gamma^{ (n-1) }}{\delta c}\frac{\delta\phantom{L}}{\delta L}-\frac{\delta\Gamma^{ (n-1) }}{\delta\tau}\frac{\delta\phantom{\psi}}{\delta\psi}+\frac{\delta\Gamma^{ (n-1) }}{\delta \psi}\frac{\delta\phantom{\tau}}{\delta \tau}-\frac{\delta\Gamma^{ (n-1) }}{\delta E}\frac{\delta\phantom{\phi}}{\delta\phi} - \frac{\delta\Gamma^{ (n-1) }}{\delta \phi}\frac{\delta\phantom{E}}{\delta E} \right. + \nonumber \\
                                 & + \left. X\frac{\delta\phantom{Y}}{\delta Y}+b\frac{\delta\phantom{\bar{c}}}{\delta\bar{c}}+B\frac{\delta\phantom{\bar{\chi}}}{\delta\bar{\chi}}+\bar{\eta}\frac{\delta\phantom{\bar{\phi}}}{\delta\bar{\phi}}+\Omega\frac{\delta\phantom{\tau}}{\delta\tau}+L\frac{\delta\phantom{E}}{\delta E}+K\frac{\delta\phantom{\lambda}}{\delta\lambda}+H\frac{\delta\phantom{Z}}{\delta Z}\right]\;,
\end{align}
and
\begin{align}
  \label{eq:linear-t-operator}
  \mathcal{T}_{\Gamma^{ (n-1) }} & \equiv \int \tr \left[\frac{\delta\Gamma^{ (n-1) }}{\delta \Omega}\frac{\delta\phantom{\psi}}{\delta \psi} + \frac{\delta\Gamma^{ (n-1) }}{\delta\psi}\frac{\delta\phantom{\Omega}}{\delta \Omega} + \frac{\delta\Gamma^{ (n-1) }}{\delta H}\frac{\delta\phantom{X}}{\delta X} + \frac{\delta\Gamma^{ (n-1) }}{\delta X}\frac{\delta\phantom{H}}{\delta H} \right. + \nonumber \\
                                 & + \left.\frac{\delta\Gamma^{ (n-1) }}{\delta L}\frac{\delta\phantom{\phi}}{\delta \phi} + \frac{\delta\Gamma^{ (n-1) }}{\delta \phi}\frac{\delta\phantom{L}}{\delta L} + \frac{\delta\Gamma^{ (n-1) }}{\delta K}\frac{\delta\phantom{B}}{\delta B} + \frac{\delta\Gamma^{ (n-1) }}{\delta B}\frac{\delta\phantom{K}}{\delta K} \right. + \nonumber                             \\
                                 & +\left.\left(\bar{c}+\bar{\eta}\right)\left(\frac{\delta\phantom{\bar{c}}}{\delta\bar{c}}-\frac{\delta\phantom{\bar{\eta}}}{\delta\bar{\eta}}\right)\right] \;.
\end{align}
Due to the recursive method of the algebraic renormalization technique, results valid for $ \Gamma^{ (1) } $ are equally valid for $ \Gamma^{ (n) } $, and \textit{vice-versa}. Additionally, the linearized Slavnov-Taylor operator~\eqref{eq:linear-st-operator} is nilpotent, and defines a cohomology which is isomorphic to the $ s $-cohomology.

The most general solution of~\eqref{eq:ward-identities} for $n=1$, represents the most general counterterm that $n$-loop radiative corrections can generate. A good way to start is via~\eqref{eq:quantum-st-identity}, due to the nilpotency of the linearized Slavnov-Taylor operators, $ {\mathcal{S}_{\Sigma}}^{ 2 } = 0 $. Its most general solution reads
\begin{equation}
  \label{eq:solution-to-quantum-st}
  \Sigma^{ (1) } = \Delta^0 + \mathcal{S}_{ \Sigma } \Delta^{ -1 }\;.
\end{equation}
$ \Delta^0 $ and $ \Delta^{ -1 } $ are the most general integrated polynomial invariant of the quantum fields and external sources, which are local, power-counting renormalizable, and a 4-form. In particular, $ \Delta^0 $ has ghost number 0, and $ \Delta^{ -1 } $ has ghost number -1. The former belongs to the $ \mathcal{S}_{ \Sigma } $-cohomology, and is given by
\begin{equation}
  \label{eq:delta0}
  \Delta^0 \equiv \int \tr \left( a_1 RR + a_2 RR^* \right) \;,
\end{equation}
where $ a_1 $ and $ a_2 $ are arbitrary renormalization parameters. The latter, also consistent with all the other Ward identities in~\eqref{eq:ward-identities}, is given by
\begin{equation}
  \label{eq:delta-1}
  \Delta^{-1} \equiv \int \tr \left( \alpha_3 Y R + \alpha_4 Y \star R + \alpha_5 Y R^* + \alpha_6 Y \star R^* + \beta \bar{ \chi } R^{\pm} \right) \;,
\end{equation}
where $ \alpha_3 $, $ \alpha_4 $, $ \alpha_5 $, $ \alpha_6 $, and $ \beta $ are also arbitrary renormalization parameters. Finally, the most general counterterm action functional, and solution to~\eqref{eq:ward-identities}, is
\begin{align}
  \label{eq:ct-action}
  \Sigma^{ (1) } & = \int \tr \left\{ \vphantom{^{\pm}} a_1 R R + a_2 R R^* + \alpha_3 \left[ X R + Y \left( D \psi + \left[ c, R \right] \right) \right] + \alpha_4 \left[ X \star R \right. + \right. \nonumber                            \\
                 & \left. + \left. Y \star \left( D \psi + \left[ c, R \right] \right) \right] + \alpha_5 \left[ X R^* + Y {\left( D \psi + \left[ c, R \right] \right)}^* \right] + \alpha_6 \left[ X \star R^* \right. + \right. \nonumber \\
                 & \left. + \left. Y \star {\left( D \psi + \left[ x, R \right] \right)}^* \right] + \beta \left[ BR^{ \pm } + \bar{ \chi } {\left( D \psi + \left[ c, R \right] \right)}^{ \pm }  \right]\right\} \;.
\end{align}

\subsection{Quantum stability}\label{sec:stability;sec:quantum}

To conclude the proof of renormalizability, one has to show the existence of finite parameters $ z_{ \Phi } $, $ z_{ G } $, $ z_{ J } $, such that
\begin{equation}
  \label{eq:multiplicative-renorm}
  \Sigma \left[ \Phi_0, G_0, J_0 \right] = \Sigma \left[ \Phi, G, J \right] + \epsilon\Sigma^{ (1) } \left[ \Phi, G, J \right] \;,
\end{equation}
where
\begin{subequations}%
  \label{eq:field-redef}
  \begin{align}
    \Phi_0 & \equiv z_{ \Phi } \Phi \;\; ; \;\; \Phi \in \left\{ \omega, c, \psi, \phi, \bar{ c }, b, \bar{ \chi }, B, \bar{ \phi }, \bar{ \eta } \right\} \;, \\
    G_0    & \equiv z_{ G } G \;\; ; \;\; G \in \left\{ g_1, \ldots, g_{10} \right\} \;,                                                                       \\
    J_0    & \equiv z_{ J } J \;\; ; \;\; J \in \left\{ \tau, \Omega, E, L, \lambda, K, Z, H, Y, X \right\} \;,
  \end{align}
\end{subequations}
is the multiplicative redefinition of all quantum fields, coupling parameters, and external sources. This requirement guarantees that no infinities are left untamed to all orders in perturbation theory. For our topological symmetry-restored phase of gravity, these $z$-factors are
\begin{subequations}%
  \label{eq:z-factors}
  \begin{align}
    z_{ X }               & = { z_{ H } }^{ -1 } \;,                                                                                                                          \\
    z_{ \omega }          & = z_{ b } = 1 \;,                                                                                                                                 \\
    z_{ g_{ 1 } }         & = 1 + \epsilon { g_{ 1 } }^{ -1 } a_{ 1 } \;,                                                                                                     \\
    z_{ g_{ 2 } }         & = 1 + \epsilon { g_{ 2 } }^{ -1 } a_{ 2 } \;,                                                                                                     \\
    z_{ g_{ 3 } } z_{ X } & = 1 + \epsilon { g_{ 3 } }^{ -1 } \alpha_{ 3 } \;,                                                                                                \\
    z_{ g_{ 4 } } z_{ X } & = 1 + \epsilon { g_{ 4 } }^{ -1 } \alpha_{ 4 } \;,                                                                                                \\
    z_{ g_{ 5 } } z_{ X } & = 1 + \epsilon { g_{ 5 } }^{ -1 } \alpha_{ 5 } \;,                                                                                                \\
    z_{ g_{ 6 } } z_{ X } & = 1 + \epsilon { g_{ 6 } }^{ -1 } \alpha_{ 6 } \;,                                                                                                \\
    z_{ B }               & = 1 + \epsilon \beta = { z_{ K } }^{ -1 } \;,                                                                                                     \\
    z_{ g_{ 7 } }         & = z_{ g_{ 8 } } =  z_{ g_{ 9 } } = z_{ g_{ 10 } }  = { z_{ H } }^{ 2 } \;,                                                                        \\
    z_{ \bar{ \phi } }    & = z_{ \tau } = z_{ L } = {z_{ \bar{ c } }}^{ 2 } = z_{ \lambda } z_{ \bar{ \chi } } = z_{ Y } z_{ Z } = {z_{ \phi } }^{ -1 } \;,                  \\
    z_{ \bar{ c } }       & = z_{ \Omega } = z_{ \bar{ \eta } } = z_{ B } z_{ \lambda } = z_{ Y } z_{ H } = z_{ E } z_{ \phi } = {z_{ \psi } }^{ -1 } = {z_{ c } }^{ -1 } \;.
  \end{align}
\end{subequations}
Clearly, the gauge field $ \omega $ does not renormalize. This is a feature of TYM due to the lack of local field equations. The physical coupling, $ g_1 $ and $ g_2 $, renormalize with the $ s $-cohomology, while the non-physical ones, $ g_3, \ldots, g_{10} $, renormalize with $ s $-boundaries. This is exactly what is to be expected.

\end{document}
