\documentclass[../main.tex]{subfiles}

\begin{document}

\section{Introduction}\label{sec:introduction}

Topological quantum field theories (TQFTs) can be interpreted as exact path integral representations of several classes of topological invariants~\cite{birmingham1991a,blau1993a,cordes1995a}. Most notably, the Gromov-Witten invariants of pseudo-holomorphic curves can be obtained from an $ n=2 $ topological $ \sigma $-model~\cite{witten1988c}. The Jones polynomials of knots and links can be obtained from $ n=3 $ Chern-Simons (CS) theory~\cite{witten1989a}. The Donaldson polynomials of smooth 4-manifolds can be obtained from $ n=4 $ Topological Yang-Mills (TYM)~\cite{witten1988d}. And so on.

By their nature, TQFTs are conformal and diffeomorphism invariants. Additionally, they are well-defined perturbative quantum field theory, often of very simple dynamics, and a finite number of degrees of freedom~\cite{brandhuber1994a,werneck1993a,sadovski2018b,sadovski2020}. For instance, gauge TQFTs tend to be exactly solvable in the perturbative regime: $ n=3 $ CS theory at 1-loop~\cite{blasi1991a,maggiore1992a,piguet1995a}, and $ n=4 $ TYM at tree-level~\cite{sadovski2017c,sadovski2018a}.

Due to these features, TQFTs represent an appealing framework to express quantum gravity ideas. Since birth, they were given the physical interpretation of an \textquote{unbroken phase of gravity}~\cite{witten1988d}. This is fully realized in low spacetime dimensions. In $n=2$, quantum GR is explicitly topological: its dynamics is given by a sum over genera. Additionally, one can construct the gravitational Labastida-Pernici-Witten model, related to Mumford-Morita-Miller classes~\cite{labastida1988b,labastida1988a}. Both approaches deal with invariant of Riemann surfaces, admit random matrix integral representations, and can be associated with a non-perturbative description of non-critical strings~\cite{douglas1990a,dijkgraaf1991a,gross1990a,witten1990a,kontsevich1992a,dijkgraaf2002a}.

As the number of spacetime dimensions grows, these fundamental relationships become foggier and foggier. In $ n=3 $, quantum GR is equivalent to CS theory only in the perturbative limit~\cite{witten1988b,witten2007a}. The success of random matrix has not being replicated by a well-behaved geometric limit for random tensor. And, the association to non-critical strings and/or conformal field theories usually requires duality conjectures. In $ n=4 $ the situation is worse. Gravity has propagating local degrees of freedom  in the bulk and any fundamental equivalence to TQFTs is lost.

%More recently, the connections between TQFTs and quantum gravity in $ n=4 $ have been explored.

Swampland arguments for a topological phase of gravity~\cite{agrawal2020a}.
Topological gravity in the very early Universe~\cite{kehagias2021a}

Topological gauge theories. Lovelock-Cartan gravity~\cite{mardones1991a,sadovski2017a}.

The idea is that classical gravity in the first order formalism can be generated from a quantum topological gauge theory~\cite{sako1997a,mielke2011a,fang2023a,}.

\end{document}

