\documentclass[../main.tex]{subfiles}

\begin{document}

\section{Introduction}\label{sec:introduction}

Topological quantum field theories (TQFTs) can be interpreted as exact path integral representations of many classes of topological invariants~\cite{birmingham1991a,blau1993a,cordes1995a}. Most notably, $ n=2 $ topological $ \sigma $-models are related to Gromov-Witten invariants of pseudo-holomorphic curves, $ n=3 $ Chern-Simons (CS) theory to Jones polynomials of knots and links, $ n=4 $ Topological Yang-Mills (TYM) to Donaldson polynomials of smooth 4-manifolds, and so on~\cite{witten1988c,witten1989a,witten1988d}.

By their nature, TQFTs are conformal and diffeomorphism invariant. Additionally, they are well-defined perturbative quantum field theory, often of very simple local dynamics, and a finite number of degrees of freedom~\cite{brandhuber1994a,werneck1993a,sadovski2018b,sadovski2020}. Gauge TQFTs, in particular, tend to be exactly solvable in the perturbative regime: $ n=3 $ CS theory at 1-loop, and $ n=4 $ TYM at tree-level~\cite{blasi1991a,maggiore1992a,piguet1995a,sadovski2017c,sadovski2018a}.

Due to these features, TQFTs represent an appealing framework to express quantum gravity ideas. In low spacetime dimensions, this program has found the most success. In $n=2$, quantum GR is explicitly topological: its dynamics is given by a sum over genera. Moreover, one can construct the gravitational Labastida-Pernici-Witten model, related to Mumford-Morita-Miller classes~\cite{labastida1988b,labastida1988a}. Both approaches deal with invariant of Riemann surfaces, admit random matrix integral representations, and can be associated with a non-perturbative description of non-critical strings~\cite{douglas1990a,dijkgraaf1991a,gross1990a,witten1990a,kontsevich1992a,dijkgraaf2002a}.

In high spacetime dimensions, these fundamental equivalences become foggier. In $ n=3 $, quantum GR is equivalent to CS theory only in the perturbative limit~\cite{achucarro1986a,witten1988b,witten2007a}. The success of random matrices has not being replicated by a well-behaved geometric limit for random tensors~\cite{gurau2024a,carrozza2024a}. And, the association to non-critical strings and/or conformal field theories usually requires duality conjectures~\cite{carlip2005a,manschot2007a,yin2008a,elshowk2012a,carlip2023a,ma2024a}.

In $ n=4 $ the situation is, of course, worse. Gravity has propagating local degrees of freedom  in the bulk, and any fundamental equivalence to TQFTs is lost. However, TQFTs still present a viable venue for an ultraviolet (UV) completion of gravity. In this scenario, GR is treated as an effective field theory, which emerges out of a TQFT phase after a symmetry break occurs~\cite{sako1997a,mielke2011a,sadovski2017a,gu2017a}. Remarkably, this proposal is able to address several issues of the very early Universe without the need of an inflationary period~\cite{agrawal2020a,kehagias2021a,fang2023a}. And, it can potentially shed light in the nature of dark matter, and hidden supersymmetry~\cite{fang2021a,raitio2024a}.

In this work, we propose an $ n=4 $ $ SO(4) $ TYM theory as a topological phase of gravity. In Section~\ref{sec:top_grav}, we show how it is able to generate not only GR, but the whole family of Lovelock-Cartan theories of gravity\footnote{This is the most general gravitational dynamics in $n=4$, which includes curvature and torsion, but excludes high-derivative terms~\cite{mardones1991a,hassaine2016a,corichi2016a}.}, after its topological symmetry is explicitly broken via the introduction of a mass scale. And, in Section~\ref{sec:quantum}, we show its Ward identities, counterterms, and quantum stability, to all orders in perturbation theory. Finally, in Section~\ref{sec:tym}, we give a brief review on TYM theories, and Section~\ref{sec:conclusions} contain our conclusions.

\end{document}

