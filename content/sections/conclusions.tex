\documentclass[../main.tex]{subfiles}

\begin{document}

\section{Conclusions}%
\label{sec:conclusions}

In this work, we proposed an $ SO \left( 4 \right) $ TYM theory, defined by~\eqref{eq:tg+sb}, as topological phase of gravity. The TYM BRST symmetry transformations~\eqref{eq:top_grav_brst}, play a pivotal role in this context. They \textquote{freeze} the local gravitational dynamics into the physically trivial $s$-boundary term~\eqref{eq:s-boundary_action}. In this topological phase, the gravitational observables that remain are related to the smoothability of spacetime.

The unprecedented $ s $-boundary term~\eqref{eq:s-boundary_action}, vital to the connection to gravity, has the potential to spoil the renormalizability features of $ SO(4) $ TYM\@. In Section~\ref{sec:quantum}, the quantum stability of the proposed model is worked out in the extended (A)SDL gauge~\eqref{eq:extended-asdlg}. Remarkably, the model remains renormalizable to all orders in perturbative theory, and contains 7 independent renormalizations~\eqref{eq:ct-action}.

Key features of a topological quantum field theories remain explicit our model. For instance, the complete set of Ward identities of traditional QTYM, in the (A)SDL gauge, is present here --- albeit corrected to account for the presence of~\eqref{eq:s-boundary_action}. We found that this strong set of symmetries implies that the gauge field does not renormalize, $ z_{ \omega } = 1 $. This result is compatible with the vanishing of $ \langle \omega(x) \omega(y) \rangle $ to all orders in perturbation theory. The latter is a known feature of traditional QTYM in the (A)SDL gauge~\cite{sadovski2017c,sadovski2018a}. This is yet another reflection of the physical content of the theory being non-local in the bulk and/or living in the boundary.

The connection between the proposed topological phase and the Lovelock-Cartan family of gravity theories is explained in Section~\ref{ssec:from_top_to_grav;sec:top_grav}. And, it boils down to a dynamical implementation of the horizontal condition, $ \psi = \phi = 0 $. This condition deforms the TYM BRST into the traditional YM BRST\@. Local observables in the bulk --- incompatible with the TYM BRST --- are allowed by the YM BRST\@. Most remarkably, the proposed implementation also identifies these bulky local degrees of freedom as gravitational.

The detailed nature of the mass parameter $ \mu $ was left open. However, several mass generation mechanisms are known in the context of non-Abelian gauge theories. These can be employed to explain the origin of $ \mu $, and to predict its current theoretical value. The viability of the model can then be tested by comparison to most current the experimental value of $ m_{ \text{P} } $, and the most current observational value of $ \Lambda^2 $, via~\eqref{eq:correspondence}.

\end{document}

