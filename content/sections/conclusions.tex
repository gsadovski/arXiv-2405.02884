\documentclass[../main.tex]{subfiles}

\begin{document}

\section{Conclusions}%
\label{sec:conclusions}

In this work, we proposed an $ SO \left( 4 \right) $ TYM theory, defined by~\eqref{eq:tg+sb}, as topological phase of gravity. The TYM BRST symmetry transformations~\eqref{eq:top_grav_brst}, play a pivotal role in this context. They \textquote{freeze} the local gravitational dynamics into the physically trivial $s$-boundary term~\eqref{eq:s-boundary_action}. In this topological phase, the gravitational observables that remain compatible with the symmetries are topological, related to the smoothability of spacetime.

The unprecedented $ s $-boundary term~\eqref{eq:s-boundary_action}, vital to the connection to gravity, has the potential to spoil the renormalizability features of $ SO(4) $ TYM\@. In Section~\ref{sec:quantum}, the quantum stability of the proposed topological phase of gravity was worked out in the extended (A)SDL gauge~\eqref{eq:extended-asdlg}. Remarkably, the model remains renormalizable to all orders in perturbative theory, and it contains 7 independent renormalizations --- given in~\eqref{eq:ct-action}.

Key features of a topological quantum field theory remain explicit in the proposed model. For instance, the complete set of Ward identities of traditional QTYM in the (A)SDL gauge is present here, albeit corrected to account for the presence of~\eqref{eq:s-boundary_action} --- these are terms containing the fields $ X $, $ Y $, $ Z $, and/or $ H $, in Section~\ref{ssec:ward_identities}. We found that this strong set of symmetries implies that the gauge field does not renormalize, $ z_{ \omega } = 1 $. This result is compatible with the vanishing of $ \langle \omega(x) \omega(y) \rangle $ to all orders in perturbation theory. The latter is a known feature of traditional QTYM in the (A)SDL gauge~\cite{sadovski2017c,sadovski2018a}. This is yet another reflection of local degrees of freedom being pure gauge in the bulk. The physical content of the theory is non-local in the bulk and/or lives in the boundary.

The connection of the topological model to the Lovelock-Cartan family of gravity theories is explained in Section~\ref{ssec:from_top_to_grav;sec:top_grav}. It boils down to a dynamical implementation of the horizontal condition, $ \psi = \phi = 0 $. This condition deforms the TYM BRST into the traditional YM BRST\@. Local observables in the bulk are incompatible with the TYM BRST, but compatible with the YM BRST\@. Most remarkably, the proposed implementation also identifies, without any extra assumptions, these local degrees of freedom in the bulk as gravitational.

The detailed nature of the mass parameter $ \mu $ was left open. However, several mass generation mechanisms are known in the context of non-Abelian gauge theories. These can be employed to explain the origin of $ \mu $, and to predict its current theoretical value. The viability of this model can then be tested by comparison to most current the experimental value of $ m_{ \text{P} } $, and the most current observational value of $ \Lambda^2 $, via the equations in~\eqref{eq:correspondence}.

\end{document}

