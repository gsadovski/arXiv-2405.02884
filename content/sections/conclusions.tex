\documentclass[../main.tex]{subfiles}

\begin{document}

\section{Conclusions}%
\label{sec:conclusions}

In this work, we proposed an $ SO \left( 4 \right) $ TYM theory, defined by~\eqref{eq:top_grav_action}, as topological phase of gravity. The TYM BRST symmetry~\eqref{eq:top_grav_brst}, plays a vital role in this context. It \textquote{freezes} the local gravitational dynamics into the BRST boundary term~\eqref{eq:s-boundary_action}. In this topological phase, all gravitational observables become topological, related to the smoothability of spacetime (Donaldson invariants).

The unprecedented BRST boundary term~\eqref{eq:s-boundary_action}, does not change the physics of $ SO(4) $ TYM, but it does change its renormalizability properties. In Section~\ref{sec:quantum}, we study them in the extended (A)SDL gauge~\eqref{eq:extended-asdlg}. Luckily, we were able to prove the model remains renormalizable to all others in perturbative theory. Additionally, key features of a topological quantum field theory remain explicit. For instance, the vectorial supersymmetry, though modified by $ X $ and $ Y $, is still present. In TYM, the vectorial supersymmetry is responsible for the vanishing of $ \langle A(x) A(y) \rangle $, to all orders in perturbation theory. This is consistent with our findings, in our topological model for gravity, of $ z_{ \omega } = 1 $

The connection to the Lovelock-Cartan family of gravity theories is possible via an explicit breaking of the BRST symmetry.
\end{document}

